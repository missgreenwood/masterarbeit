% -------------------------------------------------------------------------------------------------
%      MDSG Latex Framework
%      ============================================================================================
%      File:                  abstract.tex
%      Author(s):             Michael Duerr
%      Version:               1
%      Creation Date:         30. Mai 2010
%      Creation Date:         30. Mai 2010
%
%      Notes:                 - Place your abstract here
% -------------------------------------------------------------------------------------------------
%
\vspace*{2cm}

\begin{center}
    \textbf{Abstract}
\end{center}

\vspace*{1cm}

\noindent Ubiquitous computing und die Zunahme mobiler Endgeräte haben zu neuen Kommunikationsmustern im Internet geführt: Weg von adressbasiertem Routing und Ende-zu-Ende-Kommunikation hin zu \textit{Information-centric Networking}. Kontextzentrische soziale Netze sind ein Ansatz, dem Rechnung zu tragen. Kommunikation basiert darin nicht auf Online-Freundschaft, sondern allein auf Kontext-Ähnlichkeit. Das kontextzentrische soziale Netz AMBIENCE verwendet Bloom-Filter, um Nachrichten zu kodieren, speichern und anzufragen. Diese müssen so an einem Host organisiert werden, dass die \textit{k} nächsten Nachbarn zu einem Anfragefilter möglichst schnell und effizient gefunden werden. Die vorliegende Arbeit optimiert das mengentheoretische Problem der \textit{k}-nächsten-Nachbarn-Suche für dieses Szenario. Dazu wurde eine Indexstruktur für Bloom-Filter basierend auf einem B$^+$-Baum entwickelt. Die Bloom-Filter werden darin nach ihren Teil- und Obermengenbeziehungen organisiert, der \textit{BloomFilterTree}. Bei der \textit{k}-nächsten-Nachbarn-Suche wird nur der beste Pfad im Baum verfolgt und Teilbäume möglichst früh abgeschnitten. Kriterien für die Evaluation waren Ergebnisqualität, CPU-Zeit, Zeitkomplexität, Speicherbedarf und Aufbaukosten. Im verwendeten verwendeten Versuchsaufbau ließen sich Zeitkomplexität und CPU-Zeit der \textit{k}-nächste-Nachbarn-Suche mit dem BloomFilterTree um bis zu 68\% bzw. 87\% reduzieren. 