% -------------------------------------------------------------------------------------------------
%      MDSG Latex Framework
%      ============================================================================================
%      File:                  appendix.tex
%      Author(s):             Michael Duerr
%      Version:               1
%      Creation Date:         30. Mai 2010
%      Creation Date:         30. Mai 2010
%
%      Notes:                 - Place your appendix here
%                             - Use the same commands (`chapter', `section', ...) as in main text
% -------------------------------------------------------------------------------------------------
%
\chapter{Anhang}\label{ch:anhang}
\section{Die Klasse \texttt{BloomFilterTree}}\label{sec:BloomFilterTree.hpp}
\small{
\begin{verbatim}
//  BloomFilterTree.hpp, Judith Greif
//  Description: Header for class BloomFilterTree

#ifndef BloomFilterTree_hpp
#define BloomFilterTree_hpp

#include "BloomFilterNode.hpp"
#include "BloomFilterIndexNode.hpp"
#include "BloomFilterLeaf.hpp"

using namespace std;

class BloomFilterTree {
    
private:
    int t;                          // Order = minimum degree
    int filtersize;                 // Size of associated Bloom filters (# of bits)
    
public:
    BloomFilterNode *root;          // Pointer to root node   
    BloomFilterTree(int _t, int _s);
    ~BloomFilterTree();   
    BloomFilterNode *getRoot();
    
    // Tree management
    void traverse();
    void traverseFilters();
    double computeMinJaccard(BloomFilter *filter);
    double computeMaxJaccard(BloomFilter *filter);
    int getMinJaccardKey(BloomFilter *filter);
    BloomFilter *getMinJaccardFilter(BloomFilter *filter);
    int getMinKey();
    int getMaxKey();
    vector<BloomFilter> collectAllFilters();
    int countFilters();
    int countUnionFilters(); 
    int computeSubsetId(BloomFilter *filter);
    int computeSupersetId(BloomFilter *filter);
    bool contains(int k);
    BloomFilterNode *search(int k);
    vector<pair<int, double>>computeAllDistances(BloomFilter *filter);
    vector<pair<int, double>>computekDistances(BloomFilter *filter, int k);
    int countLeaves(); 
    
    // Measurement and comparison
    vector<pair<BloomFilter, double>> compare(BloomFilter *filter, int k);
    vector<int> compareMem();
    vector<double> compareConstrCost();
    vector<int> compareComplSimQuery(BloomFilter *filter);
    vector<int> compareComplSimQueryVec(BloomFilter *filter, int k);
    
    // Insertion
    void insert(BloomFilter *filter);
    void insertAsSets(BloomFilter *filter);
    
    // Similarity queries
    BloomFilter *simQuery(BloomFilter *filter);
    vector<BloomFilter*> simQueryVec(BloomFilter *filter, int k);
};

#endif
\end{verbatim}
}
\newpage
\section{Die Klasse \texttt{BloomFilter}}\label{sec:BloomFilter.hpp}
\small{
\begin{verbatim}
//  BloomFilter.hpp, Judith Greif
//  Description: Header for class BloomFilter

#ifndef BloomFilter_hpp
#define BloomFilter_hpp

#include <iostream>
#include <vector>
#include <cstdlib>
#include <random>
#include <math.h>
#include <string>
#include <functional>

using namespace std;

const int NUM_FILTERS = 100;
const int NUM_ELEMENTS = 50;
const int NUM_QUERYFILTERS = 10;
const int seed = rand();

class BloomFilter {
    
private:
    int id;
    int count;                  // # of elements inserted
    int size;                   // # of bits
    int d;                      // # of hash functions
    int *data;
    
public:
    BloomFilter();
    BloomFilter(const BloomFilter& fSource);
    BloomFilter(int _id, int _size);
    ~BloomFilter();   
    BloomFilter & operator = (const BloomFilter &fSource);   
    void setId(int value);
    int getId();
    int getSize();
    void setValue(int index, int value);
    int *getData();   
    void printData();
    void printArr();
    void initRandom();
    double fractionOfZeros();
    double eSize();
    BloomFilter *logicalOr(BloomFilter *filter);
    BloomFilter *logicalAnd(BloomFilter *filter);
    bool isSubset(BloomFilter *filter);
    bool isSuperset(BloomFilter *filter);
    int mySupersetCount();
    int mySubsetCount();
    int binomialCoefficient(int n, int k);
    int setOnes();
    int setZeros();
    int validOnes();
    int possibleFreeZeros();
    int possibleAddedOnes();
    double setUnion(BloomFilter *filter) const;
    double setIntersection(BloomFilter *filter) const;
    double computeAmbienceJaccard(BloomFilter *filter);
    double computeJaccard(BloomFilter *filter) const;
    double eUnion(BloomFilter *filter);
    double eIntersect(BloomFilter *filter);
    void add(string &elem);
    void increment();
    int getNumHashes();
    bool checkCorrectFillDegree();
};

#endif
\end{verbatim}
}
\newpage
\section{Die Funktion \texttt{computeSubsetId()} der Klasse \texttt{BloomFilterLeaf()}}\label{sec:computeSubsetId()}
\small{
\begin{verbatim}
int BloomFilterLeaf::computeSubsetId(BloomFilter *filter) {
    vector<pair<int, double>> subsets;
    vector<int> freeIds;
    vector<pair<int, int>> goodIds;
    BloomFilterLeaf *tmp = this;
    double jacc;
    int minId = filters[0]->getId()-1;
    int maxId;
    int optId;
    bool no_subsets = true;
    while (tmp != NULL) {
        
        // Collect all filters that filter is subset of
        for (int i=0; i<tmp->getCount(); i++) {
            if ((tmp->filters[i])->isSubset(filter)) {
                jacc = computeJaccard(tmp->filters[i], filter);
                subsets.push_back(make_pair(tmp->filters[i]->getId(), jacc));
                no_subsets = false;
            }
        }
        maxId = tmp->filters[tmp->getCount()-1]->getId()+1;
        tmp = tmp->getNext();
    }
    
    // Sort subsets by jacc distances in ascending order
    sort(subsets.begin(), subsets.end(), [](const pair<int, double> &left, 
    const pair<int, double> &right) {
        return left.second < right.second;
    });
    
    // Collect free ids
    tmp = this;
    freeIds.push_back(minId);
    freeIds.push_back(maxId);
    while (tmp != NULL) {
        for (int i=0; i<tmp->getCount()-2; i++) {
            for (int j=tmp->filters[i]->getId()+1; j<tmp->filters[i+1]->getId(); j++) {
                if (j<tmp->filters[i+1]->getId()) {
                    freeIds.push_back(j);
                }
            }
            if (tmp->getCount() < tmp->getMax()) {
                if (tmp->getNext() != NULL) {
                    int start = tmp->filters[tmp->getCount()-1]->getId()+1;
                    int last = tmp->getNext()->filters[0]->getId();
                    for (int j=start; j<last; j++) {
                        freeIds.push_back(j);
                    }
                }
            }
            
        }
        tmp = tmp->getNext();
    }
    sort(freeIds.begin(), freeIds.end(), less<int>());
    
    // If there are no subsets, return smallest free id as pair with numerical distance 0
    if (no_subsets == false) {
        
        // Determine optimal id
        // Check subsets in ascending order
        // Get next greater and smaller id
        int distNeg = subsets[0].first - minId;
        int distPos = maxId - subsets[0].first;
        for (int i=0; i<subsets.size(); i++) {
            optId = subsets[i].first;
            int j=0;
            while (freeIds[j] < subsets[i].first) {
                if (optId-freeIds[j] < optId-minId) {
                    minId = freeIds[j];
                    distNeg = optId-minId;
                }
                j++;
            }
            
            j=freeIds.size()-1;
            while (freeIds[j] > subsets[i].first) {
                if (freeIds[j]-optId < maxId-optId) {
                    maxId = freeIds[j];
                    distPos = maxId-optId;
                }
                j--;
            }
            goodIds.push_back(make_pair(minId, distNeg));
            goodIds.push_back(make_pair(maxId, distPos));
        }
        
        // Sort next smaller and greater ids by numerical distance in ascending order
        sort(goodIds.begin(), goodIds.end(), [](const pair<int, int> &left, 
        const pair<int, int> &right) {
            return left.second < right.second;
        });
    }
    else {
        goodIds.push_back(make_pair(freeIds[0], 0));
    }
    
    // Return first element
    return goodIds[0].first;
}
\end{verbatim}
}
\section{Die Funktion \texttt{simpleSimQueryVec()} der Klasse \texttt{BloomFilterLeaf}}\label{sec:simQueryVec()}
\small{
\begin{verbatim}
vector<BloomFilter*> BloomFilterLeaf::simQueryVec(BloomFilter *filter, int k) {
    vector<BloomFilter*> results;
    vector<pair<BloomFilter*, double>> distances;
    double jacc;
    
    // Collect all candidates
    // Collect own candidates
    for (int i=0; i<getCount(); i++) {
        jacc = computeJaccard(filters[i], filter);
        distances.push_back(make_pair(filters[i], jacc));
    }
    
    // Collect candidates from previous leaf
    if (prev != NULL) {
        for (int i=0; i<prev->getCount(); i++) {
            jacc = computeJaccard(prev->filters[i], filter);
            distances.push_back(make_pair(prev->filters[i], jacc));
        }
    }
    
    // Collect candidates from next leaf
    if (next != NULL) {
        for (int i=0; i<next->getCount(); i++) {
            jacc = computeJaccard(next->filters[i], filter);
            distances.push_back(make_pair(next->filters[i], jacc));
        }
    }
    
    // Sort candidates by jaccard distance in ascending order
    sort(distances.begin(), distances.end(), [] (const pair<BloomFilter*, double> &left, 
    const pair<BloomFilter*, double> &right) {
        return left.second < right.second;
    });
    
    // If tree does not hold enough filters, return all results
    if (distances.size() < k) {
        for (int i=0; i<distances.size(); i++) {
            results.push_back(distances[i].first);
        }
    }
    else {
        
        // Return first k Bloom filters from distances vector
        for (int i=0; i<k; i++) {
            results.push_back(distances[i].first);
        }
    }
    return results;
}
\end{verbatim}
}