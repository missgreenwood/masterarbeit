\chapter{Background}\label{sec:Background}
\section{Bloom Filter Operations and Variants}
\subsection{Attenuated Bloom Filter}
\cite{Sakuma2011}: 316 u. 318.
\section{Mathematic Principles}
\section{Query Processing and Index Structures in Database Systems}
To support query processing and operations in an efficient manner the internal layer of a database system relies on certain data strucures and memory methods. These are called \textit{index structures}. They organize the data to support the required operations using its \textit{indices}.

An \textit{index} (also called \textit{directory}) holds information about the structure of a file. A \textit{file} in this context refers to an entire data structure, i.e. an array, a search tree etc.. 

One can differentiate between three classes of index structures depending on the manner in which the data is organized: 
\begin{itemize}
	\item \textit{Data-organizing index structures} are used to organize the actual amount of data. They heavily rely on \textit{search trees}. 
	\item \textit{Space-organizing index structures} are used to organize the space that holds the data. They use \textit{dynamic hashing}. 
	\item \textit{Hybrid index structures} are a combination of both classes.   
\end{itemize}
\cite{Ottmann2012}
\subsection{B-Trees}	
\cite{Knuth1998}	
\subsection{R-Trees}
% "Ein R-Baum (englisch R-tree) ist eine in Datenbanksystemen verwendete mehrdimensionale (räumliche) dynamische Indexstruktur. Ähnlich wie bei einem B-Baum handelt es sich hier um eine balancierte Indexstruktur" (vgl. https://de.wikipedia.org/wiki/R-Baum)
\subsection{R*-Trees}
% "Eine beliebte R-Baum-Variation ist der R*-Baum von Norbert Beckmann, Hans-Peter Kriegel, Ralf Schneider und Bernhard Seeger. Diese Variante versucht, durch eine weiterentwickelte Split-Strategie das Überlappen von Rechtecksregionen zu minimieren. Dadurch brauchen bei einer Suchanfrage meistens weniger Teilbäume durchsucht zu werden, und die Anfragen an den Baum werden dadurch effizienter. Zusätzlich können beim Überlauf einer Seite auch Elemente neu in den Baum eingefügt werden (re-insert), was eine Aufteilung (engl. "split") (und die damit unter Umständen steigende Höhe des Baumes) vermeiden kann. Dadurch wird ein höherer Füllgrad erreicht und dadurch ebenfalls eine verbesserte Effizienz. Der resultierende Baum ist aber stets auch ein R-Baum, die Anfragestrategie ist unverändert" (vgl. https://de.wikipedia.org/wiki/R-Baum\#R.2A-Baum)
\subsection{Heaps}
% "Ein Heap (englisch wörtlich: Haufen oder Halde) in der Informatik ist eine zumeist auf Bäumen basierende abstrakte Datenstruktur. In einem Heap können Objekte oder Elemente abgelegt und aus diesem wieder entnommen werden. Sie dienen damit der Speicherung von Mengen. Den Elementen ist dabei ein Schlüssel zugeordnet, der die Priorität der Elemente festlegt. Häufig werden auch die Elemente selbst als Schlüssel verwendet. Über die Menge der Schlüssel muss daher eine totale Ordnung festgelegt sein, über welche die Reihenfolge der eingefügten Elemente festgelegt wird. Beispielsweise könnte die Menge der ganzen Zahlen zusammen mit der Kleinerrelation (<) als Schlüsselmenge fungieren. Der Begriff Heap wird häufig als bedeutungsgleich zu einem partiell geordneten Baum verstanden [...]. Gelegentlich versteht man einschränkend nur eine besondere Implementierungsform eines partiell geordneten Baums, nämlich die Einbettung in ein Array, als Heap" (vgl. https://de.wikipedia.org/wiki/Heap\_(Datenstruktur))
\section{AMBIENCE}
\cite{Werner2015}. 