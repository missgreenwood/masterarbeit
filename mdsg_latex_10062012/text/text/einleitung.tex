\chapter{Einleitung}\label{ch:einleitung}
%Einleitung
%Motivieren Sie ihre Arbeit. Warum ist diese Arbeit relevant, was erwartet die LeserInnen.
Die digitale Kommunikation hat im letzten Jahrzehnt einen rapiden Wandel erlebt. Soziale Online-Netzwerke\footnote{Der englische Begriff hierfür lautet \textit{Online Social Network (OSN)}. Die Begriffe \textit{Soziales Netz}, \textit{Soziales Online-Netz} und \textit{Soziales Online-Netzwerk} werden im Folgenden synonym verwendet.} haben an Bedeutung gewonnen und zu neuen Kommunikationsmustern im Internet geführt. Sofortnachrichtendienste und Instant-Messenger ersetzen zunehmend Kommunikationsformen wie SMS und Telefonie. Das Smartphone hat das Mobiltelefon als mobiles Endgerät fast vollständig abgelöst. Feststehende Desktop-Rechner mit einem gleich bleibenden Netzzugang sind außerhalb von Firmen und Bildungseinrichtungen rückläufig. Dagegen sind Notebooks und Tablets weiterhin auf Erfolgszug\footnote{Laut Analysen der Marktforschungsinstitute Gartner und IDC, vgl. z.B. \url{http://www.golem.de/news/pc-markt-absatz-von-pcs-geht-weiter-erheblich-zurueck-1601-118505.html}.}. 

Die Kommunikation im Internet ist jedoch nach wie vor Ende-zu-Ende- beziehungsweise adressbasiert. Das spiegelt sich im Aufbau der bestehenden sozialen Online-Netzwerke wider: Kommunikation basiert darin auf Online-Freundschaft. Mobilität und spezifischer Kontext der Mitglieder werden kaum berücksichtigt. In der Realität verlieren die Web\-brow\-ser-Schnittstellen der sozialen Netzwerke jedoch an Bedeutung. So veröffentlichte Facebook 2014 eine Studie zum Nutzerverhalten von US-Bürgern in einer Multigeräte-Welt\footnote{Vgl. \url{https://www.facebook.com/business/news/Finding-simplicity-in-a-multi-device-world}.}. Danach nutzten 60\% der Erwachsenen in den USA täglich mindestens zwei Endgeräte, knapp 25\% sogar drei Geräte. Mehr als 40\% begannen den Tag mit einem Gerät und beendeten ihn mit einem anderen. Das Smartphone ist dabei das Gerät, das am häufigsten mitgenommen wird und eine zentrale Rolle in der Kommunikation per E-Mail und in sozialen Netzen einnimmt. Das Szenario, in dem Alice vor ihrem Desktop-Rechner zu Hause oder im Büro sitzt und über die Webbrowser-Schnittstelle Nachrichten an ihren Facebook-Freund Bob schreibt, gehört demnach der Vergangenheit an. Stattdessen verwendet Alice wohl eher ein Tablet, ein Notebook und ein Smartphone und kommuniziert mit Bob je nach Aufenthaltsort und Kontext ganz unterschiedlich. 

So stellt sich die Frage nach einer Neuorientierung der sozialen Online-Netze: Nicht nur in der praktischen Umsetzung, also durch Schaffung unterschiedlicher Schnittstellen und neuer Funktionalitäten, sondern im  Sinne eines tatsächlichen Paradigmenwechsels. Ein kontextzentrisches soziales Netz basiert in seiner Struktur nicht auf Ende-zu-Ende-Kommunikation, adressbasiertem  Routing und einem gleich bleibenden Netzzugang. Kommunikation beruht allein auf Kontext-Ähnlichkeit statt auf virtueller Freundschaft. Diese Überlegungen sind z.B. in das soziale Online-Netz AMBIENCE eingeflossen. Nachrichten werden darin auf Grund von zeitlicher und räumlicher Ähnlichkeit ausgetauscht, Sender und Empfänger bleiben weitgehend anonym. Ein solches Netz erfordert eine neue Kommunikationsstruktur. Nachrichten werden nicht aktiv von einem Sender für einen spezifischen Empfänger verfasst und an ihn verschickt. Stattdessen kann ein Sender eine Nachricht verfassen und z.B. an einem WiFi-Access Point hinterlegen. Mitglieder des Netzwerks, die sich in der Nähe des Access Points aufhalten, können die dort vorhandenen Nachrichten mit gezielten Anfragen durchsuchen und die zum jeweiligen Kontext ähnlichsten Nachrichten abrufen. 

Damit stellt sich die Frage: Wie lassen sich die Nachrichten an einem Host, also z.B. an einem WiFi-Access Point, so organisieren, dass die k ähnlichsten Nachrichten möglichst schnell und effizient gefunden werden?  Wenn das soziale Netz wachsen und über den Status eines Prototypen hinaus erfolgreich sein soll, ist das von entscheidender Bedeutung. Mengentheoretisch betrachtet handelt es sich dabei um das Problem der \textit{k}-nächsten-Nachbarn-Suche, die zu einer Anfrage die k ähnlichsten Elemente einer Menge, hier bestehend aus den Nachrichten an einem Host, finden soll. 

Damit verknüpft ist die Frage nach der Nachrichtenform. Wie können Multimedia-Nach\-rich\-ten wie Bilder, Textdateien oder Links effizient, einheitlich und sicher vor unbefugtem Zugriff hinterlegt und verschickt werden? Zudem muss die Ähnlichkeit oder Unähnlichkeit von Nachrichten ermittelt werden können, d.h. der \textit{k}-nächste-Nachbarn-Suche muss ein Ähnlichkeitsmaß zu Grunde liegen. AMBIENCE verwendet dazu eine Bloom-Filter-Konstruktion. Eine Nachricht wird als Menge von Zeichenketten aufgefasst, die mit geeigneten Hashfunktionen in ein Bit-Array fester Länge eingefügt werden. Ähnlichkeit zwischen Nachrichten ist damit als Ähnlichkeit zwischen Bloom-Filtern definiert. Nachrichten werden in Form von Bloom-Filtern kodiert, gespeichert und verglichen. Als Ähnlichkeitsmaß dient die Jaccard-Distanz, mit der sich die Ähnlichkeit von Mengen beschreiben lässt. 

Die folgende Arbeit behandelt daher die Organisation von Bloom-Filtern zur effizienten \textit{k}-nächste-Nachbarn-Suche in kontextzentrischen sozialen Netzen. Das folgende Kapitel \ref{ch:hintergrund} gibt einen Überblick über mengentheoretische und probabilistische Grundlagen, verwendete Datenstrukturen und ihre Nutzung in AMBIENCE. Kapitel \ref{ch:related} gibt einen Überblick über verwandte Arbeiten und Fragestellungen. Das entwickelte Verfahren wird anschließend in Kapitel \ref{ch:implementierung} dargestellt. Kapitel \ref{ch:evaluation} vergleicht die Implementierung mit dem bisherigen, nicht optimierten Ansatz. Im abschließenden Kapitel \ref{ch:zusammenfassung} wird ein Fazit gezogen und auf mögliche zukünftige Arbeiten eingegangen. 