\chapter{Evaluation}\label{ch:evaluation}
Das folgende Kapitel dient dem Vergleich zwischen der entwickelten Datenstruktur und der bisherigen Organisation der Bloom-Filter in AMBIENCE. Abschnitt \ref{sec:datensatz} beschreibt zunächst den für die Evaluation verwendeten Datensatz. Dieser stammt nicht aus AMBIENCE, sondern die Bloom-Filter wurden wie in Abschnitt \ref{sec:umsetzung} beschrieben selbst implementiert. Der Versuchsaufbau, d.h. welche Aspekte der Indexstruktur untersucht und verglichen wurden, findet sich in Abschnitt \ref{sec:versuchsaufbau}. Die erzielten Ergebnisse werden in Abschnitt \ref{sec:ergebnisse} vorgestellt und in Abschnitt \ref{sec:interpretation} ausgewertet. 
\section{Datensatz}\label{sec:datensatz}
Obgleich nicht mit echten AMBIENCE-Daten gearbeitet wurde, wurde ein möglichst realistisches Szenario angestrebt mit folgenden Parametern: 
\begin{itemize}
	\item \textbf{Anzahl Bloom-Filter:} 100
	\item \textbf{Anzahl Anfragefilter:} 10
	\item \textbf{Anzahl eingefügte Objekte pro Bloom-Filter:} 50 
	\item \textbf{Größe der Bloom-Filter:} 256 bzw. 512 Bit
\end{itemize} 
\section{Versuchsaufbau}\label{sec:versuchsaufbau}
\section{Ergebnisse}\label{sec:ergebnisse}
\section{Interpretation}\label{sec:interpretation}
%Es lassen sich auch mehrere Bilder nebeneinander platzieren wie z.B. in Abbildung
%\ref{fig:multipic} zu sehen ist.
%\begin{figure}[hpbt]
% \centering
%  %%----start of first subfigure----
%  \subfloat[FIFO size limited to 20 entries]{
%   \label{fig:multipic:a} %% label for first subfigure
%   \includegraphics[width=0.48\linewidth]{pic1}}
%  \hspace{0.01\textwidth}
%  %%----start of second subfigure----
%  \subfloat[FIFO size limited to 30 entries]{
%   \label{fig:multipic:b} %% label for second subfigure
%   \includegraphics[width=0.48\linewidth]{pic2}}\\[0pt] % horizontal break
%  %%----start of third subfigure----
%  \subfloat[FIFO size limited to 40 entries]{
%   \label{fig:multipic:c} %% label for third subfigure
%   \includegraphics[width=0.48\linewidth]{pic3}}
%  \hspace{0.01\textwidth}
%  %%----start of fourth subfigure----
%  \subfloat[FIFO size limited to 50 entries]{
%   \label{fig:multipic:d} %% label for fourth subfigure
%   \includegraphics[width=0.48\linewidth]{pic4}}
% \caption[Observed message fractions and 95\% confidence intervals for Chord]{Observed message fractions and 95\% confidence intervals for Chord without the influence of churn. The FIFO capacity varies from 20 (\ref{fig:multipic:a}) -- 50 (\ref{fig:multipic:d}) entries (decadic steps).}
% \label{fig:multipic} %% label for entire figure
%\end{figure}
%
%\subsection{Programm Code}
%Eine elegante Möglichkeit, Programmtext einzubinden, lässt sich mit dem listings-Paket erreichen.
%Das \verb|HelloWorld| Programm aus Listing \ref{lst:hw} hat in Zeile \ref{line:hw3} übrigens einen Programmierfehler.
%\begin{lstlisting}[float=htp,caption=Hello World,label=lst:hw,language=Java, numbers=left, numberstyle=\tiny, stepnumber=2, numbersep=8pt, escapeinside={//@}{@//},backgroundcolor=\color{yellow},xleftmargin=3ex,xrightmargin=1ex]
%public class HelloWorld {
%    public static void main(String[] args) {
%        Syste.out.println("Hello, World"); //@\label{line:hw3}@//
%    }
%}
%\end{lstlisting}
