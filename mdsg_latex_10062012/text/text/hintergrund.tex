\chapter{Hintergrund}\label{ch:hintergrund}
\section{Bloom-Filter}\label{sec:bloom}
\subsection{Grundlagen}\label{sec:bloom-basics}
\textit{Attenuated Bloom Filter:} \cite{Sakuma2011}: 316 and 318\\
\textit{Counting Bloom Filter:} \cite{Fan2000}\\
\textit{Compressed Bloom Filter:} \cite{Mitzenmacher2002}
\subsection{Mathematische Prinzipien und Operationen}\label{sec:mathe}
\section{Indexstrukturen}\label{sec:indexstrukturen}
% TODO: Überarbeiten/übersetzen 
To support query processing and operations in an efficient manner, the internal layer of a database system uses specific data strucures and memory methods. These are called \textit{index structures}. They organize the data to support the required operations using its \textit{indices}.

An \textit{index} (also called \textit{directory}) of a file holds information about its structure. A \textit{file} in this context refers to an entire data structure, i.e. an array, a search tree etc.. One can differentiate between three classes of index structures depending on the manner of organization: 
\begin{enumerate}
	\item \textbf{\textit{Data-organizing index structures}} are used to organize the actual amount of data. They mostly rely on \textit{search trees}. 
	\item \textbf{\textit{Space-organizing index structures}} are used to organize the space that holds the data. They make use of \textit{dynamic hashing}. 
	\item \textbf{\textit{Hybrid index structures}} are a combination of both classes. They are based on \textit{hash trees}.   
\end{enumerate}
There are several requirements for an index structure in order to meet its purpose. 
\begin{itemize}
	\item \textit{Efficient search:} A data query on the index structure should return an answer in optimal time, i.e. the query should be directed to the page or pages that contain the queried data using as little steps as possible.
	\item \textit{Dynamic insertion, deletion and modification of data sets:} The amount of data to be organized changes over time, leading to alterations in the index structure as well. Any implementation requiring a complete reorganization of the index structure on insertion, deletion or modification of data sets is clearly unacceptable. Any of these operations may therefore only lead to local changes. 
	\item \textit{Local preservation of order:} If there are some data sets the keys of which are successors within the applied order relation (i.e. the less-or-equal relation on non-negative integers), this order should be preserved within the index structure. This holds for search trees but it does not hold for linear hashing. It is clearly of great importance regarding the application scenario in question.
	\item \textit{Efficient use of space:} This requirement is of great importance for real-world applications. So far the reference implementation \textit{AMBIENCE} has served as a proof of concept. Accordingly the number of messages, i.e. the amount of data to be queried, has been relatively small compared to a real-world scenario. Therefore the memory requirements of any index structure within the current scenario that represents the actual amount of data is unlikely to require vast amounts of memory. However, keeping in mind future application scenarios for \textit{AMBIENCE}, efficient use of space cannot be entirely discarded. 
\end{itemize}
Further requirements include \textit{feasability} and \textit{implementation cost}. Any index structure aiming at a real-world implementation such as \textit{AMBIENCE} naturally has to be feasable, so this requirement will be overlooked in the following. As this work clearly has a scholarly background, not an industrial one, the implementation cost will be disregarded as well.\\ 
\cite{Ottmann2012}
\subsection{B- und B+-Bäume}	
\cite{Knuth1998}	
\section{AMBIENCE}\label{sec:ambience}
\cite{Werner2015}. 