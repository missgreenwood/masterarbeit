\chapter{Verwandte Themen}\label{ch:related}
%Related Work / Verwandte Themen
%An dieser Stelle stellen Sie die aktuellen Stand der Technik/Forschung dar. Geben Sie eine Übersicht über Veröffentlichungen und Forschungsthemen, die sich mit ähnlichen oder sogar dem selben Thema beschäftigen. Grenzen Sie sich von anderen Veröffentlichungen ab, sodass die Frage "Warum wird diese Arbeit durchgeführt, wenn es doch schon etliche andere in diesem Bereich gibt?" beantwortet wird.
%Tip: legen Sie sich bereits während der Durchführung der Arbeit eine stichpunktartige Liste darüber an, welche Arbeiten Sie hier warum aufnehmen wollen.
% TODO Einarbeiten: 
\cite{Agarwal2006}, \cite{Byers2002}, \cite{Duerr2010}, \cite{Hellerstein1994}, \cite{Lehman1986}, \cite{Nafe2005}, \cite{Qiao2014}, \cite{Ruppel2014}, \cite{Sarwat2012}, \cite{Schnell2013}, \cite{Schoenfeld2014}, \cite{Shiraki2009}, \cite{Yang2002}, \cite{Zhang2012}, \cite{Zhu2004}, \cite{Jannink1995}.

%Es lassen sich auch mehrere Bilder nebeneinander platzieren wie z.B. in Abbildung
%\ref{fig:multipic} zu sehen ist.
%\begin{figure}[hpbt]
% \centering
%  %%----start of first subfigure----
%  \subfloat[FIFO size limited to 20 entries]{
%   \label{fig:multipic:a} %% label for first subfigure
%   \includegraphics[width=0.48\linewidth]{pic1}}
%  \hspace{0.01\textwidth}
%  %%----start of second subfigure----
%  \subfloat[FIFO size limited to 30 entries]{
%   \label{fig:multipic:b} %% label for second subfigure
%   \includegraphics[width=0.48\linewidth]{pic2}}\\[0pt] % horizontal break
%  %%----start of third subfigure----
%  \subfloat[FIFO size limited to 40 entries]{
%   \label{fig:multipic:c} %% label for third subfigure
%   \includegraphics[width=0.48\linewidth]{pic3}}
%  \hspace{0.01\textwidth}
%  %%----start of fourth subfigure----
%  \subfloat[FIFO size limited to 50 entries]{
%   \label{fig:multipic:d} %% label for fourth subfigure
%   \includegraphics[width=0.48\linewidth]{pic4}}
% \caption[Observed message fractions and 95\% confidence intervals for Chord]{Observed message fractions and 95\% confidence intervals for Chord without the influence of churn. The FIFO capacity varies from 20 (\ref{fig:multipic:a}) -- 50 (\ref{fig:multipic:d}) entries (decadic steps).}
% \label{fig:multipic} %% label for entire figure
%\end{figure}
%
%\subsection{Programm Code}
%Eine elegante Möglichkeit, Programmtext einzubinden, lässt sich mit dem listings-Paket erreichen.
%Das \verb|HelloWorld| Programm aus Listing \ref{lst:hw} hat in Zeile \ref{line:hw3} übrigens einen Programmierfehler.
%\begin{lstlisting}[float=htp,caption=Hello World,label=lst:hw,language=Java, numbers=left, numberstyle=\tiny, stepnumber=2, numbersep=8pt, escapeinside={//@}{@//},backgroundcolor=\color{yellow},xleftmargin=3ex,xrightmargin=1ex]
%public class HelloWorld {
%    public static void main(String[] args) {
%        Syste.out.println("Hello, World"); //@\label{line:hw3}@//
%    }
%}
%\end{lstlisting}

